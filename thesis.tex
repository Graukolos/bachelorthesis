\documentclass[thesis]{template/rrlab}
% !TEX root = thesis.tex

\RRLABtitle{Closed-Loop Control of the Series Elastic Actuator}
\RRLABauthor{Max Erdelmeier}
\RRLABtype{Bachelor Thesis}
\RRLABinception{1. April 2024}
\RRLABsubmission{\today}
\RRLABfirstreviewer{Prof. Dr. Karsten Berns}
\RRLABsupervisor{Oleksandr Sivak}

% needed for \begin{comment}
\usepackage{verbatim}

% needed for \includesvg
\usepackage{svg}


\begin{document}

\RRLABtitlepage{\includegraphics[width=0.8\textwidth]{assets/titlepage.jpeg}}{\today}

\RRLABsecondpage

\RRLABdeclaration{\today}

\RRLABpreface{Preface}{
    This thesis is a wonderful opportunity to thank the people who've helped me both writing it as well as on my way to it.

    First, I want to thank Prof. Karsten Berns for providing me the opportunity to write about a topic that I am deeply passionate about, that is not a given.
    Second, thanks to my supervisor Oleksandr Sivak, who helped me both with the bachelor thesis and the bachelor seminar.

    To my parents, who have always supported me in every way, I just want to say
    I love with all my heart and I'm proud to be your son.

    There are also quite a few people I would consider mentors who influenced in a big way who I am today.
    While some of them may never read this, I still want to give my thanks to them:
    Harald Pister, Henrik Roes and Clemens Schroeder
}

\RRLABabstract{Abstract}{
    \section*{English Version}
    We present a flexible motor controller architecture on a RaspberryPi.
    The goals for this implementation are an easily interchangeable control algorithm and high performance.
    We discuss several different approaches to achieve our goals: One running bare metal without an operating system, and another one on top of the linux kernel.
    For the linux kernel version we explore multiple different ways on optimizing the execution times.
    All of these will be written in the Rust programming language, a modern systems programming language with potential in the embedded world.
    This allows for an detailed analysis of Rust's intercompatibility with C.

    \section*{German Version}
    Wir stellen eine auf einem Raspberry Pi implementierte Motorsteurung vor.
    Der Regelungsalgorithmus selbst soll dabei austauschbar sein und wird für alle Beispielprogramme durch einen PID-Regler implementiert.
    Ziel ist es eine möglichst performante Implementierung zu finden, sowohl was die durchschnittlichen Reaktionszeiten, also auch die Worst-Case Laufzeiten angeht.
    Um das zu erreichen, vergleichen wir zwei grundlegend verschiedene Ansätze, einmal ein Bare-metal Programm und einmal eines das unter Linux läuft.
    Diese Implementierung nutzen wir um die Eignung der Programmiersprache Rust für embedded Programme zu untersuchen, insbesondere mit Hinblick auf die interkompatibilität mit C.
}

\RRLABcontents

% Here the topic should be introduced, and the structure, goal and scope of the thesis should be defined.
% One should focus on answering the following question: Why is the topic of the Thesis interesting for the (scientific) community?
\chapter{Introduction}
\label{chap:introduction}

\begin{comment}
FPGA expensive, slow and takes long to develop
RaspberryPi cheap and has hardware acceleration for SPI and PWM
Rust as new language competitive to C?
Bare-metal still necessary with modern Microprocessors and Real-time Linux?
What if we want to implement something different from linear control?

Closed loop control not a new problem
better results the higher the control frequency
Can we build a fast control loop while staying on Linux?
\end{comment}



\section{Closed Loop Control}
Closed loop control has been implemented in many different ways,
from mechanical structures like a centrifugal governor to specialized integrated circuits.
In this paper we will implement closed loop control of a brushless DC motor on a Raspberry Pi 4.
This effort is part of the larger goal of building a successor to CARL\cite{CARL} at RRLAB.
There are a variety of different ways the motor controllers could be implemented, the old CARL used FPGAs for example.
We chose to use a Raspberry Pi for the following reasons:
\begin{itemize}
    \item Cost when compared to a FPGA
    \item 
\end{itemize}

\section{Advantages of General Purpose Processors}

\section{}

% The main focus of this chapter should be to introduce everything that is needed to understand the further concepts of the thesis.
% This should entail everything you had to learn to understand and implement the task, and shortly cover the courses in your specialization you needed to understand the topic.
% The goal is to enable a reader from the same study program potentially from another specialization to understand the thesis.
\chapter{Background}
\label{chap:background}

\begin{comment}
Kurse aus meiner Vertiefung:
    - Prozessorarchitektur
    - Funktionales Programmieren
    - Grundlagen eingebetteter Systeme
    - OS-based programming of embedded systems

- Real Time Linux (PREEMPT-RT specifically)
- Bcm2711 hardware capabilities
- Rust general features over C (Traits, algebraic enums etc)
- Rust embedded ecosystem
- Spi
- 
\end{comment}

\section{Bare-Metal vs Operating System}
\label{sec:background:bm_vs_os}

% what does an os do? what advantages has it, what disadvantages

\subsection{Real time scheduling}
\label{sec:background:bm_vs_os:rtos}

\subsection{PREEMP-RT}
\label{sec:background:bm_vs_os:preempt_rt}

\subsection{Circle}
\label{sec:background:bm_vs_os:circle}

\section{Rust Prerequisites}
\label{sec:background:rust}

Rust is a modern systems programming language,
that claims to achieve comparable performance to languages with manual managed memory such as C and C++,
while also being "memory safe", at term we will define more clearly shortly.
Many modern languages use a garbage collector to manage memory on the heap,
which is an additional thread that periodically checks all memory allocations to see if they are still used and frees them if they are not.
This introduces an overhead, as the main execution path may be interrupted by the garbage collector and the analysis of the existing alloction takes execution time.
Manually memory managed languages on the other hand suffer from the possibility of memory errors, such as double frees and use after frees.
Typically manually memory managed languages also allow for arbitrary memory access leading to data races, reading uninitialized memory, etc.
Rust aims to be the best of both worlds, managing memory without a garbage collector, while also avoiding memory management and access errors.

\subsection{Memory managment by ownership}
\label{sec:background:rust:ownership}

To achieve this Rust employs a combination of methods, the most important one being ownership.
Ownership in this context means that every resource has exactly one variable assigned as its owner.
For our purposes resource usually means "memory on the heap",
but can mean anything that needs setup and cleanup when it is used,
such as file descriptors, sockets, or an SPI connection.
The process of binding resource initialization and destruction to the lifetime of a variable is neither new,
nor an invention of Rust,
but rather the RAII (Resource acquisition is initialization) paradigm from C++.
Since in the real world resources need to be used from multiple points, Rust introduces three ways to safely allow that:
\begin{itemize}
    \item transferring ownership, called moving in Rust
    \item lending exclusive access, using a mutable reference
    \item sharing non-exclusive access, using multiple immutable references
\end{itemize}

Of these references, there can always only be either one mutable reference or an arbitrary number of immutable references per resource.
At compile time a part of the compiler called the borrow checker enforces this by checking the lifetimes of all references.
Reference lifetimes are their own complex topic and not necessary for the rest of this thesis, so we will not go into them.

\begin{lstlisting}[language=Rust,style=colouredRust]
// create a variable a that owns a heap allocation
let a = Box::new([0; 1000]);
// move the ownership from a to b
let b = a;
// this statement would now not compile, because a does not own the allocation anymore
let c = a[0];
\end{lstlisting}

\subsection{Unsafe Rust}
\label{sec:background:rust:unsafe}

Earlier we said that we'd define "memory safety" more closely, now is the time for that.
Operations in Rust are called memory unsafe when they cause undefined behavior (UB),
where undefined behavior is defined as the following by the \cite{Rustonomicon}:
\begin{itemize}
    \item Dereferencing dangling or unaligned pointers
    \item Breaking the pointer aliasing rules of LLVMs noalias memory model
    \item Calling a function with the wrong call ABI or unwinding from a function with the wrong unwind ABI
    \item Causing a data race
    \item Executing code compiled with target features that the current thread of execution does not support
    \item Producing invalid values:
    \begin{enumerate}
        \item a bool that isn't 0 or 1
        \item an enum with an invalid discriminant
        \item a null fn pointer
        \item a char outside the ranges [0x0, 0xD7FF] and [0xE000, 0x10FFFF]
        \item a ! (! is a type used to mark unreachable or infallible values)
        \item an integer (i*/u*), floating point value (f*), or raw pointer read from uninitialized memory, or uninitialized memory in a str
        \item a reference/Box that is dangling, unaligned, or points to an invalid value
        \item a wide reference, Box, or raw pointer that has invalid metadata:
        \begin{itemize}
            \item dyn Trait metadata is invalid if it is not a pointer to a vtable for Trait that matches the actual dynamic trait the pointer or reference points to
            \item slice metadata is invalid if the length is not a valid usize
        \end{itemize}
        \item a type with custom invalid values that is one of those values, such as a NonNull that is null.
    \end{enumerate}
\end{itemize}

This comes with a problem though, in order to prevent UB from occuring,
some operations such as reading from an arbitrary memory address are not possible,
because the compiler cannot check if the would cause UB.
But in reality there are cases where it is both necessary and safe to read from specific memory addressess,
most notably when talking to memory mapped IO.

This is where the unsafe keyword in Rust comes in.
Unsafe marks sections of code where the programmer has additional tools at his disposal, but is also responsible for upholding the aforementioned guarantees.
Unsafe code that successfully upholds all the invariants is called sound, unsafe code that can produce UB is called unsound.
The additional things unsafe allows us to do are:
\begin{itemize}
    \item Dereference raw pointers
    \item Call unsafe functions (including C functions, compiler intrinsics, and the raw allocator)
    \item Implement unsafe traits
    \item Mutate statics
    \item Access fields of unions
\end{itemize}

Most of these are uninteresting for us, so we will focus only on dereferencing raw pointers and calling unsafe functions.
Dereferencing raw pointers is self-explanatory in its meaning, the interesting part here is when is it sound to do so.
The pointer must be correctly aligned and not aliased.
If the access is a write we need to make sure that we don't cause data races through an additional read or write.

Calling other unsafe functions may not seem very important at a first glance,
but this changes, when we consider that there are a few sources of unsafe functions that are not written by us.
Most important for us are C functions.
Since C code is inherently memory unsafe it is up to the programmer to check its soundness.
So calling any C code from Rust needs to be done in an unsafe block.
The second important source of unsafe functions is in the standard library.
Here are compiler intrinsics, platform specific functions and raw access to the heap allocator, which allows for a C style manual memory management.

\subsection{Rust in the embedded world}
\label{sec:background:rust:embedded}

Since we will look at Rust through the lense of an embedded program, we should discuss the ecosystem around embedded programming in Rust.
When compared to C and C++ there is a notable amount of standardization in the Rust ecosystem.
The Rust Embedded SIG (Special Interest Group) provides several key libraries that provide common interfaces for all Rust embedded software to use.
This allows for a clear distinction between hardware drivers implementing these interfaces and using a generic version of the interface.
The most important of these libraries both in general and for our uses are:
\begin{itemize}
    \item embedded-hal (interfaces for SPI, PWM, I2C, GPIO, and delays)
    \item embedded-io (interfaces for UART, USB)
    \item critical-section (interafce for uninterrupted code execution)
\end{itemize}

There are also libraries for the CAN-bus, DMA controllers, and async versions of embedded-hal and embedded-io,
but, since these are not relevant for our project and function in similar ways to embedded-hal,
we will not discuss them further.



\subsection{Typestate programming}
\label{sec:background:rust:typestate}

\section{Hardware Details}
\label{sec:background:hardware}

\subsection{SPI}
\label{sec:background:hardware:spi}

Next we take a look at the SPI protocol.
We will later use this to establish a communication link between the position encoder and the Raspberry Pi.
The name SPI stands for "Serial Peripheral Interface".
As the name already suggests it is a serial protocol for the communcition between one master and $n$ slaves.

It uses 4 lanes for communication\cite[p. 220]{SensornetzwerkeInTheorieUndPraxis}:

\begin{enumerate}
    \item CLK or SCLK - the clock generated by the master
    \item SS or CS or CE - of these lines there exists one per slave, it is used to select which slaves are currently enabled for communication. The names stand for "Slave Select", "Chip Select", "Chip Enable"
    \item MOSI or PICO - the data line which the master writes and the slaves read. The names stand for "Master out Slave in" and "Peripheral in Controller out"
    \item MISO or POCI - the data line which the slaves write and the master reads. The names stand for "Slave out Master in" and "Perihperal out Controller in"
\end{enumerate}

\begin{figure}[hp]
    \begin{center}
        \includesvg{assets/spi}
        \caption{Wiring of SPI with one master and two slaves}
        \label{fig:spi}
    \end{center}
\end{figure}

Depending on the polarity and phase of SCLK a SPI bus can be operated in four different modes.
The clock polarity is called CPOL and the clock phase CPHA.
A CPOL of $0$ means that SCLK idles at logical low and a CPOL of $1$ means that SCLK idles at logical high.
CPHA influences when data is sent.
When $CPHA = 0$ data is outputted when SCLK transitions to its idle level.
For $CPHA = 1$ data is outputted when SCLK transitions from its idle level.
The relation of CPOL and CPHA to the 4 Modes can be seen in table \ref{tab:spi_modes}.

\begin{table}[hp]
    \begin{tabular}{|l|l|l|}
        \hline
                    & CPOL = 0  & CPOL = 1     \\ \hline
        CPHA = 0    & Mode 0    & Mode 2    \\ \hline
        CPHA = 1    & Mode 1    & Mode 3    \\ \hline
    \end{tabular}
    \caption{SPI Modes}
    \label{tab:spi_modes}
\end{table}

\subsection{iC-MU}
\label{sec:background:hardware:ic-mu}


% This question should answer the question: How did others try to solve the problem, and what did they miss?
% So, why is the problem not yet solved? In short, this should define the research gap of your work.
% For this, state-of-the-Art approaches should be described.
\chapter{Related Work}
\label{chap:related_work}

\begin{comment}
"Real-Time Performance and Response Latency Measurements of Linux Kernels on Single-Board Computers"
- das Paper vergleicht kontext switch latenzen zwischen linux und preempt-rt im speziellen auf einem Rpi3
- wurde jedoch vor 6.1 gemessen

Ich sollte mir noch eine Closed-Loop-Control implementierung anschauen, idealerweise in C oder auf einem FPGA
\end{comment}

% This is the first main part of the thesis.
% Here the approach you and your supervisor decided on should be explained.
% Usually, first on a higher level and then in detail including implementation details.
% So the question that is answered is: How do you try to solve the problem at hand?

\begin{comment}
    Super loop:
    1. Fetch current control value and feedback from system
    2. Calculate new output value accordingly
    3. Adjust output (PWM)
    
    PD or PDI Control?
    
    
    Bare-metal vs
    Preempt-RT vs
    linux
\end{comment}

\chapter{Concept and Implementation}
\label{chap:concept_and_implementation}

Now for the actual implementation:\\
We will first consider the general control flow of our application, both for the bare-metal and OS-based version.
For the Linux version specifically we will then compare multiple approaches to optimize the OS both in terms of average performance as well as worst case latencies.
These different versions will later be compared to the bare-metal version in the \nameref{chap:experiments} chapter.
The bare-metal version demands a more detailed look, as it is the focus of this thesis.
As it is needed for any other step we will first look at the bootup process on the Raspberry Pi and how to produce a Rust binary that successfully boots.
From that we will continue with how the bindings to Circle were created and how and why an abstraction on top of it was written to make the underlying C code adhere to Rusts safety guarantees.

\section{The Control Loop}
\label{sec:concept_and_implementation:control_flow}

The general control flow follows the principle of a super loop.
On startup we initialize the communication hardware and from there on it's just a fetch, compute, update loop.
In the fetch step we get the current position of the actuator as well as the control value and the time passed since the last cycle.
The position is fetched through SPI from the IC-MU and the duration since the last cycle from a CPU timer.
For increased reproducability we will mock an external control value input by replacing it with a fixed signal based on timers on the Raspberry Pi.
In the compute step we apply those values to our PID-controller formula in order to get the new output value.
Last, in the update step, we update the duty cycle of the output PWM signal based on the just computed output value.

\section{Linux based version}
\label{sec:concept_and_implementation:linux}

\subsection{Approaches to minimize jitter}
\label{sec:concept_and_implementation:linux:approaches}

\begin{itemize}
    \item Default settings and hope
    \item PREEMP-RT + higher priority
    \item dedicated core
\end{itemize}

\section{Bare-Metal}
\label{sec:concept_and_implementation:bare_metal}

\subsection{rppal}


% This is the second main part of the thesis. The experiments should be suitable to validate your approach.
% Define the experimental setting, e.g. the robot or the dataset, the metrics as well as the results.
% The visualization of the results should be in a way such that it is easily understandable.
% Usually, first, the results are reported and then the effects of the results on the defined problem are discussed.
% Answer the question, what parts of the problem did you solve?
\chapter{Experiments}
\label{chap:experiments}

In order to compare our different approaches we measured the time an iteration of our control loop takes.
This allows us to compare not only the speed of each program version but also the consistency in the iteration times.

\section{Methodology}
\label{sec:experiments:methodology}

In order to make the benchmarks as comparable as possible we set a few invariants between the runs:
\begin{itemize}
    \item Each of the benchmarks is run on the exact same RaspberryPi 4B.
    \item The Pi runs at the stock core frequency of 1800MHz.
    \item The linux versions run RaspberryPi OS, a derivative of Debian Linux with the stock configuration.
    \item The kernel version is 6.1.73 for the stock and isolated core versions.
    \item The realtime version runs on 6.1.73-rt which was compiled prior on the RaspberryPi. For Instructions to replicate this kernel see the Appendix \ref{sec:appendix:realtime}.
    \item Our control loop target value is set to a constant for these tests and the IC-MU position encoder is fixed in place in order to return almost the same value each iteration.
    //TODO compiler version
\end{itemize}

\subsection{Testing the linux based versions}
The linux versions all run a common iteration function:
\begin{lstlisting}[language=Rust,style=colouredRust]
    pub fn iteration(...) -> Instant {
        // fetch step: calculate elapsed time, get new position and setpoint
        let iteration_time = last_iteration_start.elapsed();
        let iteration_start = Instant::now();
    
        let position = get_position(spi);
        let setpoint = get_setpoint();
    
        // compute step: calculate new output value
        pid.setpoint = setpoint;
        let output = pid
            .step(pid_ctrl::PidIn::new(position, iteration_time.as_secs_f64()))
            .out;
    
        // update step: output the new value over PWM
        pwm.set_duty_cycle(output).unwrap();
    
        iteration_start
    }
\end{lstlisting}

Time measurment is done here through Rusts stdlib Instant and Duration types.
On linux these compile to the clock\_gettime syscall in order to get microsecond accurate time.
Because the RaspberryPi 4 does not yet have a real time clock like the Raspberry Pi 5 linux uses the clock interrupts to measure the time.
To execute the benchmarks the criterion framework for Rust was used as it allows easy configuration of the amount of samples,
runs an unmeasured three second warmup loop of the benchmark beforehand and automatically generates plots and statistics data from the results.

For the isolated core version we use the cset python program in order to easily manipulate the linux kernels cpuset subsystem.
This allows us with few commands to isolate one core from all running processess and run our program on it, without being disturbed by core local interrupts.
Global interrupts such as spinlocks when running a syscall can however still halt our execution flow.

We can run cset from inside of our program with the current process id this way:
\begin{lstlisting}[language=Rust,style=colouredRust]
    Command::new("sudo")
        .args([
            "cset",
            "shield",
            "--cpu=3",
            "--kthread=on",
            &format!("--pid={}", std::process::id()),
        ])
        .spawn()
        .expect("Could not start cset binary")
        .wait()
        .expect("cset did not exit successfully");
\end{lstlisting}

For the realtime version we needed a kernel that can boot with the Raspberry Pi bootloader,
have the out-of-tree kernel modules for spi and pwm and support fully preemtive scheduling.
In order to get that combination, the official sources for the Rpi kernel were patched with the fitting version of the PREEMP\_RT patchset and compiled.

To achieve this Step 1 is to get all the required build dependencies.
\begin{lstlisting}[language=bash, breaklines]
    sudo apt update && sudo apt install build-essential flex bison libssl-dev bc
\end{lstlisting}

Step 2 is to get the kernel sources and patch them with the correct PREEMPT-RT patchset.
\begin{lstlisting}[language=bash, breaklines]
    wget https://github.com/raspberrypi/linux/archive/refs/tags/stable_20240124.tar.gz
    wget https://cdn.kernel.org/pub/linux/kernel/projects/rt/6.1/older/patch-6.1.73-rt22.patch.xz
    tar -xf stable_20240124.tar.gz
    cd linux-stable_20240124
    xzcat ../patch-6.1.73-rt22.patch.xz | patch -p1
\end{lstlisting}

Step 3 is to generate the kernel config, we use the provided default config for the Raspberry Pi 4 and only set the PREEMPT\_RT config value to enable full kernel preemption.
\begin{lstlisting}[language=bash, breaklines]
    make bcm2711_defconfig
    ./scripts/config -e PREEMPT_RT
    make olddefconfig
\end{lstlisting}

Step 4 is to actually compile the kernel, this takes about 3 hours on all 4 cores of the Raspberry Pi 4.
\begin{lstlisting}[language=bash, breaklines]
    make -j4 Image.gz modules dtbs
\end{lstlisting}

Step 5 is to actually install the files.
\begin{lstlisting}[language=bash, breaklines]
    sudo make modules_install
    sudo cp arch/arm64/boot/dts/broadcom/*.dtb /boot/firmware/
    sudo cp arch/arm64/boot/dts/overlays/*.dtb* /boot/firmware/overlays/
    sudo cp arch/arm64/boot/dts/overlays/README /boot/firmware/overlays/
    sudo cp arch/arm64/boot/Image.gz /boot/firmware/kernel8.img
\end{lstlisting}

At the end simply reboot to load the new kernel.

\subsection{Testing the bare-metal version}
Benchmarking the bare-metal version is a bit more involved than the linux based versions.
For measuring times Circle's CTimer::GetClockTicks64() is used. This in turn returns the amount of ticks of a 1MHz oscillator on the Rpi.
Similar to criterion in the linux based versions we let the loop run for some period before measuring
in order to avoid any latency spikes because things are not yet copied to the caches.
We set this to 10000 iterations, which as we will later see in the results translates to roughly 3-4 seconds of warmup.

The second difficulty with the bare-metal version is getting the results out of the RaspberryPi.
We need to either transmit the data over a connection such as ethernet, spi, i2c to another PC or save it to a filesystem on removable storage.
Since we are running bare-metal that means we need a driver for one of these options.
Because we are already using the Circle library for Spi and it provides simple access to a USB mass storage device with a FAT32 file system we will be using it to save the results.

The process for this is equivalent to how we already used the SPI and PWM drivers in the \nameref{chap:concept_and_implementation} chapter,
so we will only skim over the most important parts.

In our wrapper.hpp we need to include "circle/fs/fat/fatfs.h" and "circle/usb/usbhcidevice.h"
\begin{lstlisting}[language=C++]
    #include "circle/fs/fat/fatfs.h"
    #include "circle/usb/usbhcidevice.h"
\end{lstlisting}

In our main.rs we initialize these devices and save the measured times as a csv file.
\begin{lstlisting}
    let mut usb_hci =
        ffi::CXHCIDevice::new(&mut interrupt_system, &mut timer, false, 0, null_mut());
    let mut filesystem = ffi::CFATFileSystem::new();


    ((*usb_hci._base._base.vtable_).CUSBController_Initialize)(&mut usb_hci._base._base, true);

    for _ in 0..10000 {
        ...
    }
    const N: usize = 200;
    let mut times = [0; N];
    for time in times.iter_mut() {
        ...
        *time = ffi::CTimer::GetClockTicks64() - iteration_start;
        ...
        iteration_start = ffi::CTimer::GetClockTicks64();
        ...
    }

    let partition = device_name_service.GetDevice(c"umsd1-1".as_ptr(), true);
    filesystem.Mount(partition);

    let file = filesystem.FileCreate(c"times.csv".as_ptr());
    let mut buffer = String::from("iteration,elapsed_time_us\n");
    for (n, time) in times.iter().enumerate() {
        buffer.push_str(&format!("{},{}\n", n, time));
    }
    let buffer = alloc::ffi::CString::new(buffer).unwrap();

    filesystem.FileWrite(
        file,
        buffer.as_ptr() as *const c_void,
        buffer.count_bytes() as u32,
    );
    filesystem.FileClose(file);
    filesystem.UnMount();
\end{lstlisting}

\section{Results}
\label{sec:experiments:results}

% first table with average, std dev, min, max
\begin{table}
    \label{tab:measurments}
\end{table}

% second distribution graphs

% third profiling?

\section{Interpretation}

\begin{comment}
Measure iteration time over x iterations.
Derive average, minimal and maximal iteration times.
Look at variance 
\end{comment}


% Here the approach should be discussed related to the initial scope and goal.
% Answer the questions: What did you solve and especially what did you not solve? Give hints for potential extensions as future work.
\chapter{Conclusion}
\label{chap:conclusion}

We initially set out to create a flexible motorcontroller on a RaspberryPi,
with the idea being that different control schemes could easily be implemented in software and tested.
To reach this goal we looked at several different approaches,
both from a detailed look at the implementation as well as the resulting performance characteristics.
For all of the approaches we used the programming language Rust,
whose role in the embedded world is still an area of active research.
While a lot of research focuses on how Rusts powerful static analysis can be applied to embedded systems,
we examined the interaction of Rust with C/C++ libraries.

The different versions we developed were:
\begin{enumerate}
    \item A bare-metal version using the Circle C++ library to access peripherals
    \item A version running on the standard linux kernel using the Rust native rppal library for peripheral access
    \item A variant of the linux version that had one core reserved for it to avoid being interrupted or having to wait for a core to be available.
    \item A second variant of the linux version that exchanges the default linux kernel with one patched with a realtime patchset
\end{enumerate}

When testing these for performance, we found two main results.
For one, the linux based versions all exhibited regular iteration time spikes, of up to 2.5 milliseconds,
while the regular run time was more around 40 microseconds.
In this behavior, the linux based versions did not differ significantly from each other,
so for this workload our approaches at optimizing the thread allocation were unsuccessful.
Second, the bare-metal version performed about 10 times faster than the linux based versions,
all while not having the latency spike problem.

Regarding the usage of Rust we found that most of the added complexity of adding another language is not in the language interaction itself,
but rather in the surrounding tooling such as managing multiple compilers, linkers and build systems to all work together.
This can be seen both as a positive as well as a negative.
The positive side is that the language tools such as bindgen for generating the bindings to the C code are in a good shape both from a usability and stability perspective,
while the negative side is the added build system complexity might deter people that are just trying to code the actual program.

This leaves several directions in which this research could be expanded in future work.
Writing the bare-metal program in C for a performance comparison between Rust with a C++ library and pure C++ would be a logical follow-up to the embedded Rust work.
And on a different route, an in-depth analysis of why and how the latency spikes for the linux version occur could yield results how to achieve comparable performance to the bare-metal version.


\RRLABbibliography{literatur}

\RRLABindex

\appendix

\chapter{Appendix}
\label{chap:appendix}

\section{Unabridged code}

\subsection{bare-metal .cargo/config.toml}
\label{sec:appendix:code:bare-metal:config}

\lstinputlisting[]{code/bare-metal/.cargo/config.toml}

\subsection{bare-metal build.rs}
\label{sec:appendix:code:bare-metal:build}

\lstinputlisting[language=Rust,style=colouredRust]{code/bare-metal/build.rs}

\subsection{bare-metal Cargo.toml}
\label{sec:appendix:code:bare-metal:cargo}

\lstinputlisting[]{code/bare-metal/Cargo.toml}

\subsection{bare-metal src/ffi.rs}
\label{sec:appendix:code:bare-metal:ffi}

\lstinputlisting[language=Rust,style=colouredRust]{code/bare-metal/src/ffi.rs}

\subsection{bare-metal src/main.rs}
\label{sec:appendix:code:bare-metal:main}

\lstinputlisting[language=Rust,style=colouredRust]{code/bare-metal/src/main.rs}

\subsection{bare-metal wrapper.hpp}
\label{sec:appendix:code:bare-metal:wrapper}

\lstinputlisting[language=C++]{code/bare-metal/wrapper.hpp}

\subsection{Rust native HAL build.rs}
\label{sec:appendix:code:hal:build}

\lstinputlisting[language=Rust,style=colouredRust]{code/hal/build.rs}

\subsection{Rust native HAL Cargo.toml}
\label{sec:appendix:code:hal:cargo}

\lstinputlisting[]{code/hal/Cargo.toml}

\subsection{Rust native HAL link.x}
\label{sec:appendix:code:hal:link}

\lstinputlisting[]{code/hal/link.x}

\subsection{Rust native HAL src/critical\_section.rs}
\label{sec:appendix:code:hal:critical_section}

\lstinputlisting[language=Rust,style=colouredRust]{code/hal/src/critical_section.rs}

\subsection{Rust native HAL src/entry.rs}
\label{sec:appendix:code:hal:entry}

\lstinputlisting[language=Rust,style=colouredRust]{code/hal/src/entry.rs}

\subsection{Rust native HAL src/gpio.rs}
\label{sec:appendix:code:hal:gpio}

\lstinputlisting[language=Rust,style=colouredRust]{code/hal/src/gpio.rs}

\subsection{Rust native HAL lib.rs}
\label{sec:appendix:code:hal:lib}

\lstinputlisting[language=Rust,style=colouredRust]{code/hal/src/lib.rs}

\end{document}
