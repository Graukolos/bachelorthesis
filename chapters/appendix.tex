\chapter{Appendix}
\label{chap:appendix}

\section{Setting up a Raspberry Pi 4 with a realtime kernel}
\label{sec:appendix:realtime}
This is a short guide on how to reproduce the realtime kernel that we used for all realtime experiments.

To do this you need a Raspberry Pi 4 with a 64Bit Raspberry Pi OS installed (doesn't matter if it's the full desktop version or just a TTY one).

Step 1 is to get all the required build dependencies.
\begin{lstlisting}[language=bash, breaklines]
    sudo apt update && sudo apt install build-essential flex bison libssl-dev bc
\end{lstlisting}

Step 2 is to get the kernel sources and patch them with the correct PREEMPT-RT patchset.
\begin{lstlisting}[language=bash, breaklines]
    wget https://github.com/raspberrypi/linux/archive/refs/tags/stable_20240124.tar.gz
    wget https://cdn.kernel.org/pub/linux/kernel/projects/rt/6.1/older/patch-6.1.73-rt22.patch.xz
    tar -xf stable_20240124.tar.gz
    cd linux-stable_20240124
    xzcat ../patch-6.1.73-rt22.patch.xz | patch -p1
\end{lstlisting}

Step 3 is to generate the kernel config, we use the provided default config for the Raspberry Pi 4 and only set the PREEMPT\_RT config value to enable full kernel preemption.
\begin{lstlisting}[language=bash, breaklines]
    make bcm2711_defconfig
    ./scripts/config -e PREEMPT_RT
    make olddefconfig
\end{lstlisting}

Step 4 is to actually compile the kernel, this takes about 3 hours on all 4 cores of the Raspberry Pi 4.
\begin{lstlisting}[language=bash, breaklines]
    make -j4 Image.gz modules dtbs
\end{lstlisting}

Step 5 is to actually install the files.
\begin{lstlisting}[language=bash, breaklines]
    sudo make modules_install
    sudo cp arch/arm64/boot/dts/broadcom/*.dtb /boot/firmware/
    sudo cp arch/arm64/boot/dts/overlays/*.dtb* /boot/firmware/overlays/
    sudo cp arch/arm64/boot/dts/overlays/README /boot/firmware/overlays/
    sudo cp arch/arm64/boot/Image.gz /boot/firmware/kernel8.img
\end{lstlisting}

At the end simply reboot to load the new kernel.
