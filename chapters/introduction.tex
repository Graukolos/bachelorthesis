% Here the topic should be introduced, and the structure, goal and scope of the thesis should be defined.
% One should focus on answering the following question: Why is the topic of the Thesis interesting for the (scientific) community?
\chapter{Introduction}
\label{chap:introduction}

\begin{comment}
FPGA expensive, slow and takes long to develop
RaspberryPi cheap and has hardware acceleration for SPI and PWM
Rust as new language competitive to C?
Bare-metal still necessary with modern Microprocessors and Real-time Linux?
What if we want to implement something different from linear control?

Closed loop control not a new problem
better results the higher the control frequency
Can we build a fast control loop while staying on Linux?
\end{comment}



\section{Closed Loop Control}
Whenever you have a stateful system there are two ways of controlling it.
In Open Loop Control only the should-be signal influences the output value.
In Closed Loop or Feedback Control the output is composed of the should-be value and a feedback signal that measures the actual state of the system.
In many real-world applications Closed Loop Control is used for its ability to quickly reach a new target value, while keeping oscillation and overshoot to a minimum.

For these real-world applications it has been implemented in many different ways,
from mechanical structures like a centrifugal governor to specialized integrated circuits.

In this paper we will implement closed loop control of a brushless DC motor on a Raspberry Pi 4.
This effort is part of the larger goal of building a successor to CARL\cite{CARL} at RRLAB.
There are a variety of different ways the motor controllers could be implemented, the old CARL for example used FPGAs.
We chose to use a Raspberry Pi, because it provides us with some advantages and exciting opportunities.

\section{Advantages of Closed Loop Control on General Purpose Processors}
When compared to the FPGA approach used in CARL Single Board Computers (SBCs) such as a Raspberry Pi are cheaper and easier to program.
This comes at a cost however, typically FPGAs are faster at high speed IO operations, which is an important port of a Closed Loop Controller.
Fortunately the Raspberry Pi 4 has special purpose hardware for SPI and PWM IO which should help us alleviate this disadvantage.

When compared to ready built PID ICs a Raspberry Pi is more expensive, consumes more power and must be programmed vs only settings a few variables.
But it comes with only major upside for a research project such as ours:
It is not bound to a traditional PID controller. If the need arises to use a different control scheme or compute additional tasks in parallel, the Raspberry Pi can!

In addition using a Raspberry Pi provides an opportunity to use the Rust programming language and validate some of its claims.

\section{A new programming language for embedded systems?}
The embedded world is dominated by C and C++.
These languages come with a few well known downsides however:
\begin{itemize}
    \item race conditions are easy to create
    \item uninitialized memory can accidentally be read
    \item management of resources is manual, which typically leads to memory leaks 
\end{itemize}

Rust is a relatively new systems programming language that claims to be well suited to the embedded space while avoiding the aforementioned problems.
