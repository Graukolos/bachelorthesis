% The main focus of this chapter should be to introduce everything that is needed to understand the further concepts of the thesis.
% This should entail everything you had to learn to understand and implement the task, and shortly cover the courses in your specialization you needed to understand the topic.
% The goal is to enable a reader from the same study program potentially from another specialization to understand the thesis.
\chapter{Background}
\label{chap:background}

\begin{comment}
Kurse aus meiner Vertiefung:
    - Prozessorarchitektur
    - Funktionales Programmieren
    - Grundlagen eingebetteter Systeme
    - OS-based programming of embedded systems

- Real Time Linux (PREEMPT-RT specifically)
- Bcm2711 hardware capabilities
- Rust general features over C (Traits, algebraic enums etc)
- Rust embedded ecosystem
- Spi
- 
\end{comment}

\section{OS-related stuff}
\subsection{Real time scheduling}

\section{Rust related stuff}
\subsection{Rust - A modern systems programming language}

\subsection{Memory managment by ownership}

\subsection{Unsafe Rust}

\subsection{Rust in the embedded world}

\subsection{Typestate programming }

% alternateive Idee Hardware details
\section{Hardware Prerequisites}

\newpage
\subsection{SPI}
\label{sec:background:spi}
Next we take a look at the SPI protocol.
We will later use this to establish a communication link between the position encoder and the Raspberry Pi.
The name SPI stands for "Serial Peripheral Interface".
As the name already suggests it is a serial protocol for the communcition between one master and $n$ slaves.

It uses 4 lanes for communication\cite[p. 220]{SensornetzwerkeInTheorieUndPraxis}:

\begin{enumerate}
    \item CLK or SCLK - the clock generated by the master
    \item SS or CS or CE - of these lines there exists one per slave, it is used to select which slaves are currently enabled for communication. The names stand for "Slave Select", "Chip Select", "Chip Enable"
    \item MOSI or PICO - the data line which the master writes and the slaves read. The names stand for "Master out Slave in" and "Peripheral in Controller out"
    \item MISO or POCI - the data line which the slaves write and the master reads. The names stand for "Slave out Master in" and "Perihperal out Controller in"
\end{enumerate}

\begin{figure}[hp]
    \begin{center}
        \includesvg{assets/spi}
        \caption{Wiring of SPI with one master and two slaves}
        \label{fig:spi}
    \end{center}
\end{figure}

Depending on the polarity and phase of SCLK a SPI bus can be operated in four different modes.
The clock polarity is called CPOL and the clock phase CPHA.
A CPOL of $0$ means that SCLK idles at logical low and a CPOL of $1$ means that SCLK idles at logical high.
CPHA influences when data is sent.
When $CPHA = 0$ data is outputted when SCLK transitions to its idle level.
For $CPHA = 1$ data is outputted when SCLK transitions from its idle level.
The relation of CPOL and CPHA to the 4 Modes can be seen in table \ref{tab:spi_modes}.

\begin{table}[hp]
    \begin{tabular}{|l|l|l|}
        \hline
                    & CPOL = 0  & CPOL = 1     \\ \hline
        CPHA = 0    & Mode 0    & Mode 2    \\ \hline
        CPHA = 1    & Mode 1    & Mode 3    \\ \hline
    \end{tabular}
    \caption{SPI Modes}
    \label{tab:spi_modes}
\end{table}

\subsection{iC-MU}
%