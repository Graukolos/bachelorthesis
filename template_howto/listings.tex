\documentclass[a4paper,10pt]{article}
\usepackage[utf8x]{inputenc}

%%%%% include and setup of LISTINGS package
\usepackage{listings}
\lstset{language=java, basicstyle=\ttfamily, breakautoindent, breaklines, keywordstyle=\textbf}

%%%%% title
\title{Listings-Test}
\author{Daniel Schmidt}

\begin{document}
\maketitle

%%%%% abstract
\begin{abstract}
Diese Datei soll das listings-Paket testen. Dabei geht es vor allem um die Zeilennummern, die in der Version 1.4 anscheinend nicht korrekt (oder anders als in 1.3) gesetzt werden. Hier fangen diese immer bei 1 an, auch wenn 5-10 als Bereich definiert wurde.
\end{abstract}

%%%%% section
\section{Listings}

\subsection{Beispielscode}
Anbei der Code, den man auch hier lesen kann: \lstinline$for (int i = 1; i < 10; i ++) { ... }$, obwohl beides nix miteinander zu tun hat. Hier also der ganze Code:

\begin{lstlisting}[numbers=left, frame=single]
// loop                         // line 1
for (int i = 1; i < 10; i ++)   // line 2
{                               // line 3
  // comment                    // line 4
  int j = i;                    // line 5
  System.out.println(++j);      // line 6
}                               // line 7
// and now somthing more...     // line 8
int dummy = 10;                 // line 9
if ( dummy > 8 )                // line 10
{                               // line 11
  System.out.println(dummy);    // line 12
}                               // line 13
\end{lstlisting}

\subsection{Problem: falsche Zeilennummer}
Und hier mittels \texttt{\verb|firstline=2|} und \texttt{\verb|lastline=7|} begrenzt:

\begin{lstlisting}[numbers=left, frame=single, firstline=2, lastline=7]
// loop                         // line 1
for (int i = 1; i < 10; i ++)   // line 2
{                               // line 3
  // comment                    // line 4
  int j = i;                    // line 5
  System.out.printl(++j);       // line 6
}                               // line 7
// and now somthing more...     // line 8
int dummy = 10;                 // line 9
if ( dummy > 8 )                // line 10
{                               // line 11
  System.out.println(dummy);    // line 12
}                               // line 13
\end{lstlisting}

\subsubsection*{Lösung: Angabe der ersten Zeilennummer}
Wenn man zusätzlich noch \texttt{\verb|firstnumber=2|} angibt, so erreicht man damit durchaus den gewünschten Effekt, dass die echten Zeilennummern mit den angegebenen übereinstimmen. Auf die nächste folgende Zeilennummer (hier: \thelstnumber) abhängig vom vorangegangenen Listings kann man übrigends mittels \texttt{\verb|\thelstnumber|} zugreifen.

\begin{lstlisting}[numbers=left, frame=single, firstline=2, lastline=7, firstnumber=2]
// loop                         // line 1
for (int i = 1; i < 10; i ++)   // line 2
{                               // line 3
  // comment                    // line 4
  int j = i;                    // line 5
  System.out.printl(++j);       // line 6
}                               // line 7
// and now somthing more...     // line 8
int dummy = 10;                 // line 9
if ( dummy > 8 )                // line 10
{                               // line 11
  System.out.println(dummy);    // line 12
}                               // line 13
\end{lstlisting}

\subsection{Problem: Zeilennummern bei mehreren Bereichen}
Problematisch wird es aber letzten Endes bei der Variante mit mehreren ausgewählten Bereichen \texttt{\verb|linerange={2-2,4-6}|}. In diesem Beispiel fallen auch noch die Zeilen mit den geschweiften Klammern weg:

\begin{lstlisting}[numbers=left, frame=single, linerange={2-2,4-6}]
// loop                         // line 1
for (int i = 1; i < 10; i ++)   // line 2
{                               // line 3
  // comment                    // line 4
  int j = i;                    // line 5
  System.out.printl(++j);       // line 6
}                               // line 7
// and now somthing more...     // line 8
int dummy = 10;                 // line 9
if ( dummy > 8 )                // line 10
{                               // line 11
  System.out.println(dummy);    // line 12
}                               // line 13
\end{lstlisting}

\subsubsection*{Lösung: Einzelne Listings mit separaten Zeilenangaben und negativem 'vspace' dazwischen}
Wenn man sich die Rahmen wegdenkt (bzw. extern drumlegt) kann man über zwei separate \texttt{\verb|\begin{lstlisting}|}-Befehle und einem entsprechenden Setzen der \texttt{\verb|firstnumber|} den gewünschten Effekt erhalten. Dabei wird allerdings noch ein \texttt{\verb|\vspace*{-0.55cm}|} benötigt, um die beiden Listings näher zusammenzurücken:

\begin{lstlisting}[numbers=left, frame=single, linerange={2-2}, firstnumber=2]
// loop                         // line 1
for (int i = 1; i < 10; i ++)   // line 2
{                               // line 3
  // comment                    // line 4
  int j = i;                    // line 5
  System.out.printl(++j);       // line 6
}                               // line 7
// and now somthing more...     // line 8
int dummy = 10;                 // line 9
if ( dummy > 8 )                // line 10
{                               // line 11
  System.out.println(dummy);    // line 12
}                               // line 13
\end{lstlisting}
\vspace*{-0.55cm}
\begin{lstlisting}[numbers=left, frame=single, linerange={4-6},firstnumber=4]
// loop                         // line 1
for (int i = 1; i < 10; i ++)   // line 2
{                               // line 3
  // comment                    // line 4
  int j = i;                    // line 5
  System.out.printl(++j);       // line 6
}                               // line 7
// and now somthing more...     // line 8
int dummy = 10;                 // line 9
if ( dummy > 8 )                // line 10
{                               // line 11
  System.out.println(dummy);    // line 12
}                               // line 13
\end{lstlisting}

\noindent Noch andere Vorschläge?!?

\end{document}
