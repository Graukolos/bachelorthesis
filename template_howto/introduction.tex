%%%%%%%%%%%%%%%%%%%%%%%%%%%%%%%%%%%%%%%%%%%%%%%%%%%%%%%%%%%%
%% introduction.tex
%%
%% Kapitel: Introduction
%% Autor: Axel Vierling (axel.vierling@cs.rptu.de)       
%% Autor: Jochen Hirth (j_hirth@informatik.uni-kl.de)       
%% Autor: Tobias Luksch (luksch@informatik.uni-kl.de)       
%% Datum: Juli 2003                                         
%%                                                          
%% Letzte Änderung Dezember 2023
%%%%%%%%%%%%%%%%%%%%%%%%%%%%%%%%%%%%%%%%%%%%%%%%%%%%%%%%%%%%
\chapter{Introduction}
\label{chap:intro}

%%%%%%%%%%%%%%%%%%%%%%%%%%%%%%%%%%%%%%%%%%%%%%%%%%%%%%%%%%%%
\section{Motivation}
\label{sec:intro:motiv}
The Robotics Research Lab in the Department of Computer Science at the \RPTU offers a variety of opportunities for student work, including thesis, projects, seminars, and internships. All these assignments require the creation of a document that should meet the standards of a scientific work. In order to make it easier for students to meet these requirements, the research group provides a \LaTeX-template that realizes an appropriate layout.

%%%%%%%%%%%%%%%%%%%%%%%%%%%%%%%%%%%%%%%%%%%%%%%%%%%%%%%%%%%%
\section{Objective of the Document}
\label{sec:intro:goal}
The goal of this documentation is to provide students with a guide on creating scientific documents. It addresses specific features of the provided \LaTeX-class as well as some fundamental formalities of scientific works. It demonstrates the correct creation and referencing of images or tables and how to compile a bibliography. This document itself serves as an example and can be used as a template and starting point for individual work. It is worth noting that this is not an introduction to \LaTeX, nor does it provide guidance on installing \LaTeX-distributions. For these purposes, relevant literature (for example, \cite{Lamport95}, \cite{Goossens96}) or websites (\cite{WWWDante}, \cite{WWWMikTex} or \cite{WWWPospiech}) are recommended. Furthermore, it should be noted that the writing style of this document is more formal in many parts (though not consistently throughout) than necessary. This choice is intended to give the reader an impression of the style expected in a scientific work. Therefore, the author asks for forgiveness for the lack of informality and humorous digressions.

%%%%%%%%%%%%%%%%%%%%%%%%%%%%%%%%%%%%%%%%%%%%%%%%%%%%%%%%%%%%
\section{Structure of the Document}
\label{sec:intro:structure}
Chapter~\ref{chap:class} initially deals with the \LaTeX class \latexklasse provided by the Robotics Research Lab. It explains the required format for the main document and the available macros.

Following that, Chapter~\ref{chap:floats} provides instructions on creating specific document elements. It covers images and tables as well as formulas and citations.

Chapter~\ref{chap:structure} offers brief guidance on the general structure of a scientific work. It should only be considered as a suggestion, as a comprehensive introduction would exceed the scope of this documentation.

Chapter~\ref{chap:addendum} solely serves to illustrate a possible structure of the main document, which can exist as a single file or incorporate additional files, such as individual chapters, as separate entities.

%%% Local Variables: 
%%% mode: latex
%%% TeX-master: "howto"
%%% End: 
